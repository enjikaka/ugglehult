\chapter{Första dagen vid Ugglehult.}

Nu några ord om den första dagen. Den gav oss just den
rütta bilden af vår värd.

Det gällde att i Smålands skogar leta ut platser för en
planerad stor fiskodlingsanstalt — den skall bli den största i Norden
och en av de största i världen. Bildad var en förening, som med
stöd av sju hushållningssällskap skulle grunda en fiskeriskola för
riket och driva försöksfiske i sjöar och försöksodling i dammar.
Offentliga anslag av 25,000 kr. årligen äro att beräkna. En
tysk fiskmästare är anställd.

\begin{figure}[H]
    \centering
    \includesvg[width=0.3\linewidth]{images/02.svg}
    \caption{Carl M. Peterson.}
\end{figure}

Detta vore något att få
för vår ort, tänkte herr
Peterson. Där är redan
fiskdammskultur. Herr Peterson själv har
fiskdammar, och en annan
person i orten driver fiskodling i
stor utsträckning. Han säljer
karp och sutare till Tyskland
för tusentals kronor årligen.

Föl' att nå sitt mål,
fiskodlingsanstaltens förläggande
till Aneboda socken, har herr
Peterson offrat otroligt med
tid och arbete. Han hade
visserligen icke egna marker
att upplåta, men var ej
mindre intresserad för det. Här
var ju en möjlighet att höja
bela orten. I stora volymer samlade han brev och beräkningar
rörande fiskodling.

På middagen skulle till Ugglebult komma två representanter
för föreningen, en direktör och en kapten. Tidigt på morgonen
var herr Peterson i rörelse för att ordna arbetet vid sin gård.
Sedan for ban ut till skogsmarkerna och till markägarna för att
avsluta de sista förberedelserna.

Så snart ban hemkommit hade han att göra den
ansträngande färden för andra gången, nu i sällskap med
förenings-representanterna och den andre besökaren, som även fick följa
med.


Sedan vi rest sä, långt vi kunde bomma efter hästar
vandrade vi till fots genom skogarna med herr Peterson i spetsen.
Marschen varade i timtal, och herr Peterson tycktes känna varje
Stig och varje snår, varje rännil och varje förändring i markens
nivå. Varmt och övertygande framhöll han den hittills
värdelösa markens lämplighet för det ändamål, varom här var fräga,
och kunde icke nog prisa Vår Herre, som överallt kring myrarna
lagt vallar av grus, alldeles som om han menat att världens
största fiskeriförsöksstation här skulle komma till stånd. Han såg
redan i andanom att utefter grusåsarna, där vi marscherade, snart
skulle anläggas vägar, där sju landshövdingar skulle fara fram
för att inspektera vad som blivit gjort. Och när marschen var
till ända hade herr Peterson i skogarna visat ut plats för 19
fiskdammar, tillsammans omkring 140 tunnland, ävensom lämpliga
byggnadsplatser för det blivande institutet.

När vi återkommo till Ugglehult var där anordnad stor
festmiddag. Herr Peterson förberedde oss själv på att huvudrätten
skulle bliva någonting riktigt fint. Och en synnerligen smakfull
anrättning var det också. När måltiden led mot sitt slut fingo
vi försöka gissa, vad slags kött det kunde vara. Men det kunde
vi icke. Förvåningen blev allmän, då det upplystes att det var
kanin. Det hade ingen trott att kanin kunde smaka på det
sättet. Och herr Peterson hade fått tillfälle att övertyga sina
gäster om en annan av sina satser, nämligen att kaninskötseln
kan bliva en sak av stor betydelse och att kaninkött är en
läckerhet, om det anrättas så som fru Peterson förstår den konsten.
Bordet upptog för övrigt bröd, som fru Peterson nyss bakat,
sylter av bär från Smålands skogar och dricka som mor Sara bryggt.

Klockan 5 återvände föreningens representanter till stationen.
Men för herr Peterson var hans arbetsdag i fiskodlingens tjänst
därmed icke slut. På kvällen hade han kallat bönderna samman
i en gård framme vid Aneboda kyrka. För tredje gången for
ban den ganska långa vägen dit. Nu skulle han för institutets
räkning arrendera eller köpa den mark som behövdes. Det blev
ett besvärligt göra, ej mindre besvärligt på sitt sätt än marschen
i skogen. Men med herr Petersons förmåga att föra såväl ord
som penna klarades även denna del av uppgiften. Omsorgsfullt
genomgingos kontrakt med tjogtals punkter. Och när klockan var
nio på kvällen hade herr Peterson på hand såväl marker som rännilar, tre vattenfall och en kvarn samt lämpliga
byggnadsplatser för institutet.

Bättre kan man svårligen använda en arbetsdag.
