\chapter{Till Ugglehult.}

Vårt fosterland, vårt vida land! Dag och natt rulla vi
fram å ilande tåg och se dock icke dina gränser. Ett par timmar
är nog att flytta oss över andra betydande länder. Vårt vackra
land, vårt rika land! Vilka skatter du rymmer i åkrar och ängar,
i skogar och berg, i sjöar och forsar!

Här är rikedom nog, bara vi vore dugande nog. Men där
brister det allt icke så litet. Vi behöva dem som väcka oss och
lära oss och driva oss fram. Och sådana äro oss också givna.

\begin{figure}[H]
    \centering
    \includesvg[width=0.6\linewidth]{images/01.svg}
    \caption{Verandan vid Ugglehult med studiebesökare.}
\end{figure}

Vi äro långt inne i Smålands skogar, där man tror att
ingenting annat finnes än stenar och ljung. Här ligger Ugglohult,
Carl M. Petersons gård, ett småbruk, en bondgård, eller en liten
herrgård — det är ej gott att veta vad man rättast skall kalla den.

Tvåvåningshuset vid gården har en veranda av snidat, trä,
från 7ilken vi taga den första överblicken. Rakt ned ifrån gården
till den täcka sjön Stråken med dess holmar och uddar leder en
av granhäckar kantad väg. Vattnet höjer sig i terasser. Hitom
Stråken finnes en betydligt högre vattenyta. Det är en av herr
Petersons konstgjorda fiskdammar, på vars yta ståtliga gäss
simma omkring. Närmare gården se vi flockar av höns och kalkoner.

Dörrarna öppnas och ett glatt välkommen ljuder oss till
mötes.

Ja, nog är det en liten herrgård, det intrycket stärkes, då vi
hunnit inomhus. Här är så prydligt ordnat i den stora förstugan
och i alla rummen. Här är så rymligt och rikt. Till höger är
matsalen, där bordet oupphörligt dukas för resande gäster, till
vänster herr Petersons arbetsrum och bibliotek. I husets bakre
avdelning har fru Peterson sina arbetsrum, köket och
hushållsmejeriet. En trappa upp är gästrum, bokhållares och elevers rum
och en föreläsningssal med en svart tavla, som för tillfället upptager
ett schema för utfodring av kor och en grundritning till en fiskdamm.

Här är en ständig ström av folk och post. Flera hundra
långväga resande besöka årligen gården. Dessutom kommer
oupphörligt folk från orten. För dem har herr Peterson inrättat ett
privat postkontor, som han utan kostnad sköter. Han hämtar
deras post från stationen och sorterar ut den i amerikanska fack,
till vilka en var har sin egen nyckel. Det Ur betecknande för
gården och ägaren.

Till Ugglehult kommer ortens befolkning med sitt smör, som
Br format i vackra kilostycken och stämplas med nummer från
varje gård, varefter det skickas till staden och röner en strykande
åtgång. Posten, som nu kommer, innehåller ett brev från
handlanden, som bara klagar över att han icke kan få smör nog.
Till Ugglehult kom, medan vi voro där, den ena lantbrukaren
efter den andra och hämtade utsäden, som lantbrukare på andra
håll ej ens känna till namnet, eller konstgödsel, som av herr
Peterson var avpassad för de olika jordarnas behov.

I förstugan läses i stora bokstäver å ett anslag: Tid är
pengar. Detta hindrar dock icke, att herr Peterson och hans fru
offra sin tid åt andra. Den första dagen, då vi voro där, ägnade
herr Peterson varje timme åt ett allmännyttigt företag i orten,
den andra dagen sysselsatte han sig uteslutande med oss.
