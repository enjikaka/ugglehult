\chapter{Herr Peterson visar sin gård.}

Den följande dagen ägnade han som sagt åt oss.

Hun började sin förevisning vid gödselstaden, där en av hans
män höll på att täcka gödseln med jord. Han visade
pressvatten-brunnen, som endast upptager pressvatten från gödseln, ty urinen
uppsuges med
torvströ i gödselrännor,
som sakna avlopp.
Han visade
torvströtorkningen, med
vilken en kvinna från
orten var sysselsatt.
Dagligen äro ett
dussin personer i full
verksamhet vid den
lilla gården.

Han förde oss
in i ladugården, där
han har flera stora
djur än tunnland, ty
Ugglehult har 17
tunnland åker, men
de stora djuren äro
20. Han visade de
rymliga båsen eller
rättare kättarna, där
djuren stå två och
två. Han visade
krubbor och bindslen
och anordningarna för djurens drickesvatten. Vi sågo koii Åsa,
något gammal nu, men på sin tid mjölkande 5,000 liter eller
mer än någon annan av de 783 korna i kontrollföreningen. Vi
sågo kamrater till henne, som hade en otroligt hög mjölkproduktion och fetthalt. Vi fingo veta att samtliga djuren voro
uppdragna av herr Peterson själv och samtliga tuberkelfria. Vi fingo
höra att han runt omkring i landet utsänt rena ayrshiretjurar.

Vi se anslagen på ladugårdens väggar om kornas vätta våvd.

"Vi föras därnäst in i det lilla bokhållerikontoret vid sidan om
ladugården. Här står mjölkvågen, på vilken varje mjölkspann
väges, och här ligger mjölkboken, där varje resultat genast
antecknas. Såväl mjölkvåg som mjölkbok, liksom så mycket annat
vid Ugglehult, har konstruerats av herr Peterson eller under hans
ledning och sprides härifrån ut över landet.

Herr Peterson visar oss rotfruktskällaren vid ingången till
ladugården. Ett anslag på dörren anger att den rymmer 166
tunnor. Här är ännu ett stort lager av härliga morötter.
Ro-vorna ha gått för länge sedan, betor och kålrötter ligga i stukor
på fältet. Vi marschera upp på höskullen, där hövågen hänger
på en takbjälke och där hö är uppvägt för envar av veckans
dagar. För de äldre djuren finnas sju höbås och för ungdjuren
sju. Vi titta in i kraftfodermagasinet, där kakor av alla slag
blandas i vederbörliga mängder, sedan de passerat kakkrossen. Här
intill ha vi sädesmagasinet med ett stort lager av guldregnshavre,
som säljes som utsäde till höga pris. Säden är upplagd i luftiga
lådor, som envar rymma tre fjärdedels hektoliter, och i stället
för att omskyiflas tömmes den från låda till låda. Här är ock
frörenseriet, där ortens klöverfrö sorteras, varefter de bästa
kvalitéerna säljas för goda pris. En sak till sir att
anteckna från detta departemang. I ett förrum finna vi två
träställningar, en på vardera sidan, och säckar hänga på båda. Vid
den ena ställningen läses på ett anslag: »Denna plats endast för
söndriga säckar». Vid den andra: »Denna plats endast lör hela
Säckar».

Vi besöka stallet, där anslag inom glas och ram angiva
regler för hästens vård. Läsaren skall framdeles finna dem i
dessa blad. Vi gå till hönshuset, som skötes av mor Sara enligt
en hönsbok, som hon själv upprättat i ett handskrivet exemplar.
Vi bliva framdeles i tillfälle att se vad den innehåller. Därinne
i hönshuset finnas endast de reglementerade sju möblerna, och
mat-hoarna stå mycket riktigt i förstugan. Av rotfrukterna därinne
på hönshusets golv är ej stort annat än skalen kvar. Noggrann
räkenskap föres här liksom överallt annars. Gården säljer till
mor Sara varje handfull säd, som hon ger åt hönsen, och köper

av henne varje ägg, som användes i hushållet. Räkenskaperna
för ett år visa, att hon fått 456 kr. för sitt arbete, sodan allt
vad hönsen förtärt blivit betalt. Hon hade dä mellan 50 och 60
höns, 100 kalkoner och 7 eller 8 gäss. Vi se ett upplag av
gåsägg, som äro värda 1: 50 stycket. Vi ge rundsävslådor för
försäljning av ägg, delade i avdelningar för halva tjog, och av
dessa avdelningar tar man med endast så många, som man för
tillfället behöver.

Vi gå vidare. Vi passera bigården, där ramkupor äro
uppställda i dubbla rader, och vi komma till samlingen av biredskap,
som är ordentligt uppställd i ett rum för sig. Vi se honung i
burkar och honung i askar. De senare äro fyllda av bien själva
och därmed färdiga för servering på bordet.

Svinhuset är vår nästa station. De praktiska och renliga
svinhoar, som herr Peterson vid sina föreläsningar visar i modell,
finnas här i verkligheten. Hit hör också potatiskokaren, i vilken
herr Peterson på en gång kokar avfallspotatis för hela vintern,
varefter densamma nedgräves i jorden. Därpå föras vi till gässens
och kalkonernas enldn bostäder. I ett särskilt litet hus finnas
apparater och syror för provning av mjölkens fetthalt. Plit
komma mjölkprov från ladugårdarna i trakten, 6ch omkring 8,000
sådana mjölkprov bestämmas och bokföras årligen.

Sedan vandra vi litet längre bort, till källan med fisksumpen,
i vilken gäddor hållas på lager, och till stranden av sjön, där
fiske bedrives, till ängen som vattnas med överrissling nu på
våren, till den mångåriga gräsmarken, till fältet med gråvial och
till åkerns olika skiften. Det är verkligen sant att stenmurarna
kring dessa åkrar på sina ställen äro 20 fot breda. Här ligga
stenmassor i oräknade tusental av lass. Täckdiken äro framdragna
mellen stenarna i bottnen, och på sina ställen har berget blivit
genomsprängt för beredande av avlopp. Överallt ligga dock stenar
kvar för att sprängas efter hand.

Vi beskåda trädgården med mer än 100 fruktträd av olika
sorter, bärbuskar, jordgubbsland och rabarberplanteringar. Vi se
drivbänkar och köksväxtsängar. Vi vandra bort åt gårdens andra
sida, till mossen, där torvströ skördas på ytan och bränntorv
upptages från de djupare lagren. Vi gå från fiskdamm till fiskdamm.
»Här är något av det vackraste ni kan få skåda», säger herr
Peterson och låter oss se ned i en vinterdamm, där ett stim av
karpar och sutare håller på att vakna efter vinterdvalan. Han

sticker ned sin liåv och slänger en vacker sutarehona npp på
marken. Därefter sticker han ned håven igen och får upp en ny
sutare. Och så bär han de två till en nyanlagd damm, där ingen
fisk förut finnes. Här kastar han ut dem i vattnet, där de
blixtsnabbt försvinna, och här skola de efter kort tid lämna tusental
av yngel, som efterhand fördelas på nya dammar. Det blir mat
för de fina borden i Berlin, dit karpen och sutaren säljas for
1: 50 kilo — för konsumenterna kostar den sedan 3 kr.

Sist besöka vi skogen, där herr Petersons hushållningssystem
kanske firar sina vackraste triumfer. Han har ett par hundra
tunnland skogsmark, som han systematiskt sköter. Det enda
redskap han häi-7id behöver är en yxa. I stället för de eländiga
buskarna på andra håll och en skogsgrund täckt av ogräs ser
man här ett pelarvalv av raka stammar och nere vid marken
ingenting annat än skogens mossa. På bestämda tider avverkas
ett norr om fröskogen liggande och för vinden skyddat område.
Och här uppträder snart ungskogen av sig själv, självsådd från
fröskogen i söder. Alla träd, vilkas kronor solen icke når, avverkas
för gårdens behov.

Detta är herr Petersons hela skogshushållningslära. En tjäder,
som just nu flyger upp, erinrar oss emellertid, att den har en
punkt till. Den innehåller att villebrådet skall skyddas och skördas.
Herr Peterson har därom närmare utlåtit sig i tryckta skrifter,
som han utgivit.